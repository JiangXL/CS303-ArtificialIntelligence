\documentclass[conference,compsoc]{IEEEtran}

% *** CITATION PACKAGES ***
%
\ifCLASSOPTIONcompsoc
  % IEEE Computer Society needs nocompress option
  % requires cite.sty v4.0 or later (November 2003)
  \usepackage[nocompress]{cite}
\else
  % normal IEEE
  \usepackage{cite}
\fi


\usepackage{amsmath}
\usepackage{algorithm, algorithmic}
\renewcommand{\algorithmicrequire}{\textbf{Input:}}
\renewcommand{\algorithmicensure}{\textbf{Output:}}
\usepackage{array}
\usepackage{url}
% correct bad hyphenation here
\hyphenation{op-tical net-works semi-conduc-tor}


\begin{document}
\title{Project 2: Capacitated Arc Routing Problems }

% author names and affiliations
% use a multiple column layout for up to three different
% affiliations
\author{\IEEEauthorblockN{Yuejian Mo  11510511}
\IEEEauthorblockA{Department of Biology\\
Southern University of Science and Technology\\
Email: 11510511@mail.sustc.edu.cn}}

% make the title area
\maketitle

\IEEEpeerreviewmaketitle

\section{Preliminaries}
This project is an implementations of Path-Scanning and Dijkstra's algorithm to
solve the capacitated arc routing problems(CARP). CARP is the most typical form
of arc routing problem, which has many application in real world, such as urban
waste collection, post delivery. \cite{1} In the case of waste collection, some
stresses of city has waste, cars with limit capacitated will carry the these
waste start from depot and return again when car is full. A solution with less
cost will be perfect.

CARP can be represented formally as follows: a mixed graph $G=(V,E,A)$, with a
set of  vertices denoted by $V$, set of edges denoted by $E$ and a set of arcs
denoted by $A$, is given. There is a central depot vertex $dep \ni V$, where a
set of vehicles are based. A subset $E_R \supseteq E$ composed of all the edges
required to be served and a subset . The objective of CARP is to determine a
set of routes for the vehicles to serve all task with minimal costs while
satisfying\cite{1}: 
a) Each route must start and end at $dep$
b) The total demand serviced on each route must not exceed the vehicle's
capacity $Q$; 
c) Each task must be served exactly once(but the corresponding edge can be traversed more
than once).

CARP is NP-hard, various heuristics and meatheuristics are used, such as
Augement-Merge, Path-Scanning. Here I successfully implemented Path-Scanning.

% Add figure here

\subsection{Software}
This project is written by Python 3.7 with editor Atom and Vim. Numpy library
and sys library are used.

\subsection{Algorithm}
Using Dijkstra's algorithm to calculate the closest pathway between two vertex.
Path-Scanning is used to find out task sequence for CARP. I defined three group
of functions in order to find out optimal service sequence, including one
function to generate cost graph and demand graph between vertex and vertex
from provided txt file, two functions to generate shortest distance and pathway
dictionary by Dijkstra algorithm, other are control flow and output format
function. Function $Better$ break the balanced status.


\section{Methodology}
CARP is NP-hard. Path-Scanning provide a reasonable heuristic method to reduce
the total cost. Firstly, edge with required was copy to a new list. Car begin
its service from depot(vertex 1). Then the shortest pathway away from now
service vertex will be choose as new service vertex while the car is not full
and here still are required edges. Repeat above produce to clean up all required
edges.

\subsection{Representation}
Some main data are maintain during process: \textbf{capacity},\textbf{graph\_dm}
, \textbf{graph\_ct}.
Others data would be specified inside functions, shortest\_dist should noticed.

\begin{itemize}
  \item \textbf{capacity}: The car's capacity, generated from input file.
  \item \textbf{graph\_dm}: The graph of edge with demand and their demand,
  generated from input file.
  \item \textbf{graph\_ct}: The graph of edge with cost and their cost,
  generated from input file.
  \item \textbf{shortest\_dist}: A dictionary of shortest distance and pathway
  between two vertexes. 
\end{itemize}


\subsection{Architecture}
Here list all functions in given code:
\begin{itemize}
    \item \textbf{generateGraph}: Generate cost and demand graph from input file.
    \item \textbf{dijkstra}: Calculate all vertexes closest distance and
    pathway away from specify source.
    \item \textbf{genDijkstraDist}: Generate each two vertexes closest distance
    and pathway from \textbf{dijkstra} function.
    \item \textbf{better}: Second rule to choose one vertexes from two vertex. 
    \item \textbf{pathScan}: Path-Scanning algorithm to generate serve sequence. 
    \item \textbf{s\_format}: Standard output function. 
    \item \textbf{\_\_name\_\_}: Main control function. 
\end{itemize}
The CARP\_solver would be executed in test platform.


\subsection{Detail of Algorithm}
Here describes some vital functions.
\begin{itemize}
    \item \textbf{generateGraph}: Generate global cost and demand graph from
    input file.
    \begin{algorithm}[H]
     \caption{generateGraph}
     \begin{algorithmic}[1]
     \renewcommand{\algorithmicrequire}{\textbf{Input:}}
     \renewcommand{\algorithmicensure}{\textbf{Output:}}
     \REQUIRE $input\_file\_name$
     \ENSURE $ $
     \STATE open $input\_file\_name$ as $file$ \COMMENT{open file and read line
     by line}
     \STATE $capacity \leftarrow $file$.readline$
     \STATE \COMMENT{Read each line information until arrive edge information}
     \FOR {each edge $e$}
       \STATE split each line string into an array $line$ with 4 elements.
       \IF{ ($line[3]$ larger than 0)} 
          \STATE $graph\_dm$[($line$[0], $line$[1])]= $line$[3]
          \STATE $graph\_dm[(line[1], line[0])]= line[3]$
          \COMMENT {Only add edge to dictionary $graph\_dm$when edge has
          demand. Both of two direction will be created.}
       \ENDIF
       \STATE $graph\_ct[(line[0], line[1])]= line[2]$
       \STATE $graph\_ct[(line[1], line[0])]= line[2]$
     \ENDFOR
     \end{algorithmic}
   \end{algorithm}

   \item \textbf{dijkstra}: generate closest distance and pathway away from source
     \begin{algorithm}[H]
     \caption{dijkstra}
     \begin{algorithmic}[2]
     \renewcommand{\algorithmicrequire}{\textbf{Input:}}
     \renewcommand{\algorithmicensure}{\textbf{Output:}}
     \REQUIRE $source$
     \ENSURE  $dist$, $prev$
     \STATE create vertex set $Q$
     \STATE create distance set $dist$
     \STATE create path set $prev$
     \FOR { each vertex $v$ in $graph\_ct$ }
        \STATE $dist[v] \leftarrow INFINITY$
        \STATE $prev[v] \leftarrow UNDEFINED$
        \STATE add $v$ to $Q$
     \ENDFOR
     \STATE $dist[source] \leftarrow$ 0
     \WHILE{($Q$ is not empty)}
        \STATE $u$ $\leftarrow$ vertex in $Q$ with min $dist[u]$
        \STATE remove $u$ from $Q$
        \FOR {for each neighbor $v$ in $u$}
           \STATE $alt \leftarrow dist[u] + graph\_ct(u,v)$
           \IF {alt larger than $dist[u]$}
              \STATE $dist[v] \leftarrow alt$
              \STATE $prev[v] \leftarrow u$
           \ENDIF
        \ENDFOR
     \ENDWHILE
     \RETURN $dist$, $prev$
     \end{algorithmic}
     \end{algorithm}

  \item \textbf{genDijstraDist}: generate closest distance and pathway between
  two vertex into a dictionary $shortestDist$
    \begin{algorithm}[H]
     \caption{genDijstraDist}
     \begin{algorithmic}[3]
     \renewcommand{\algorithmicrequire}{\textbf{Input:}}
     \renewcommand{\algorithmicensure}{\textbf{Output:}}
     \REQUIRE $ $
     \ENSURE  $shortestDist$
     \STATE create dictionary $shortestDist$
     \FOR {each edge's first vertex in $graph\_dm$ as $source$ }
       \STATE $dist$, $prev$ $\leftarrow$ dijkstra($source$) 
       \FOR {each edge's first vertex in $graph\_dm$ as $target$ }
         \STATE $shortestDist[source, target] \leftarrow dist$
       \ENDFOR
     \ENDFOR
     \RETURN $shortestDist$
     \end{algorithmic}
     \end{algorithm}
 
 \item \textbf{better}: choose better in pointers 
     \begin{algorithm}[H]
     \caption{better}
     \begin{algorithmic}[4]
     \renewcommand{\algorithmicrequire}{\textbf{Input:}}
     \renewcommand{\algorithmicensure}{\textbf{Output:}}
     \REQUIRE $now$, $pre$
     \ENSURE  $Ture$ or $False$
     \RETURN $graph\_dm[now] < graph\_dm[pre]$
     \end{algorithmic}
     \end{algorithm}
 

   \item \textbf{pathScan}: Path Scanning algorithm to determine serve routes
     \begin{algorithm}[H]
     \caption{pathScan}
     \begin{algorithmic}[4]
     \renewcommand{\algorithmicrequire}{\textbf{Input:}}
     \renewcommand{\algorithmicensure}{\textbf{Output:}}
     \REQUIRE $shortest\_dist$
     \ENSURE  $R$, $cost$
     \STATE create successive routes $R$
     \STATE copy required edge from $graph\_dm$ to $free$ 
     \STATE $k, cost \leftarrow 0$
     \WHILE { $free$ is not empty}
       \STATE $k \leftarrow k+1$
       \STATE $cost\_k, load\_k \leftarrow 0 $
       \STATE rest successive routes $R\_k$ each time 
       \STATE serve from origin vertex: $source \leftarrow 1$
       \WHILE { $free$ in not empty}
         \STATE rest shortest distance: $d \leftarrow \infty$ 
         \STATE rest candidate serve edge: $e\_candidate \leftarrow -1$
         \FOR {each edge $e$ in $free$}
            \IF {$load\_k + graph\_dm[e] \le capacity$}
               \STATE $dist\_now \leftarrow shortest\_dist[source, e.start]$
               \IF {$dist\_now \le d$}
                 \STATE $d \leftarrow dist\_now$
                 \STATE $e\_candidate \leftarrow e$
               \ELSIF {$dist\_now = d$ $\cap$ better($e$,$e\_candidate$ )}
                 \STATE $e\_candidate \leftarrow e$
               \ENDIF
            \ENDIF
         \ENDFOR
         \IF {$e = \infty$}
            \STATE BREAK
         \ENDIF
         \STATE add $e\_candidate$ to $R\_k$
         \STATE $load\_k = load\_k + graph\_dm[e\_candidate]$
         \STATE $cost\_k = cost\_k + graph\_ct[e\_candidate]$
         \STATE $i = e\_candidate.end$
         \STATE remove $e\_candidate$ and its opposite edge from free
       \ENDWHILE
       \STATE add back home distance: $cost\_k = cost\_k + shortest\_dist[i,1]$
       \STATE add $R\_k$ to $R$
       \STATE $cost = cost + cost\_k$
     \ENDWHILE
     \RETURN $R$, $cost$
     \end{algorithmic}
     \end{algorithm}


\end{itemize}


\section{Empirical Verification}
Empirical verification is confirmed in public test platform. Both right output
format and reasonable routes are produced.

\subsection{Design}
Dijkstra's algorithm return successfully shortest distance and pathway between
two vertex. Path-Scanning run correctly. However, Path-Scanning don't provide
most optimal routes in spite of the size of test data. Function $better$

\subsection{Data and data structure}
Dictionaries and lists are used widely rather than matrix. Because the input graph
is sparse, dictionary are always used to store graph. Global variable
$graph\_dm$ and $graph\_ct$ are dictionary. Lists always store routes and edges
information.

\subsection{Performance}
Following table show different performance with different dataset.

\begin{center}
   \begin{tabular}{| l | l | l |}
   \hline
    Dataset  & Run Time(s) & Cost \\ \hline
    gdb1     & 0.65        & 370  \\ 
    gdb10    & 0.63        & 309  \\
    val1A    & 0.68        & 212  \\
    val4A    & 0.92        & 450  \\
    val7A    & 4           & 334  \\
    egl-e1-A & 0.96        & 4201 \\
    egl-s1-A & 5.02        & 6383 \\
   \hline
   \end{tabular}
\end{center}

\subsection{Result}
Solutions are accepted for all data set test. Larger data set require more time.

\subsection{Analysis}
Because Path-Scanning is kind of greedy algorithm with heuristics, most of
solutions are not optimal. The space cost is close to $O(n)$. First time cost
$O(n^2)$ is in building Dijkstra's distance, which only run once. Path-Scanning
without function $Better$ just cost $O(n)$. Total cost time most vary on 
function $Better$. For example, if $Better$ just return random result, it will
cost less time than comparison method. Although complex and well-design $Better$
cost more time, it's solution is less-cost-routes.

\section*{Acknowledgment}
Thanks TA Yao Zhao who explain question and provide general method to solve it.
And I also thanks for Kebing Sun discuss algorithm and point the output
formation error.

\bibliographystyle{IEEEtran}
\begin{thebibliography}{1}
\bibitem{1}
Ke Tang, Yi Mei, and Xin Yao, “Memetic Algorithm With Extended Neighborhood
Search for Capacitated Arc Routing Problems,” IEEE Transactions on Evolutionary
Computation, vol. 13, no. 5, pp. 1151–1166, Oct. 2009.

\end{thebibliography}

% that's all folks
\end{document}


