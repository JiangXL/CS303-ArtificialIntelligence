\documentclass[conference,compsoc]{IEEEtran}

% *** CITATION PACKAGES ***
%
\ifCLASSOPTIONcompsoc
  % IEEE Computer Society needs nocompress option
  % requires cite.sty v4.0 or later (November 2003)
  \usepackage[nocompress]{cite}
\else
  % normal IEEE
  \usepackage{cite}
\fi


\usepackage{amsmath}
\usepackage{algorithm, algorithmic}
\renewcommand{\algorithmicrequire}{\textbf{Input:}}
\renewcommand{\algorithmicensure}{\textbf{Output:}}
\usepackage{array}
\usepackage{url}
% correct bad hyphenation here
\hyphenation{op-tical net-works semi-conduc-tor}


\begin{document}
\title{Project 1: Gomoku}

% author names and affiliations
% use a multiple column layout for up to three different
% affiliations
\author{\IEEEauthorblockN{Yuejian Mo  11510511}
\IEEEauthorblockA{Department of Biology\\
Southern University of Science and Technology\\
Email: 11510511@mail.sustc.edu.cn}}

% make the title area
\maketitle


\IEEEpeerreviewmaketitle

\section{Preliminaries}
This project is doing some simple implementations on Gomoku. Gomoku, also called 
Five in a Row, is an abstract strategy board game. It is traditionally played 
with Go pieces (black and white stones) on a Go board, using 15x15 of the 19x19
grid intersection.\cite{1} Gomoku has less complicity than Go, which has encouraged
Deep Mind to develop powerful AI engine AlphaGo in 2016. So Gomoku will be a
better particle for beginner. Design of Gomoku AI in this project try to
follow the minMax tree with alpha-beta punching, which powered AI Deep Blue to
challenge top chess player successfully in the first time.

\subsection{Software}
This project is written by Python 3.7 with editor Atom and Vim. Numpy library
is being used.


\subsection{Algorithm}
Depth-first search in the form of minMax tree with alpha-beta punching is used.
The method being used is recursion. I defined three group of functions in order
to find out next chess position from input chess board. Including given
functions, here are nine functions in go.py, and one is for testing.


\section{Methodology}
Different chess layout present different chance to win. For example, live-four
and double-three must lead to success. So I evaluate the score of possible
layout in final level of DFS. For every layout, I scan each line in vertical,
horizontal and diagonal to calculate score and sum three together valued as
whole chess layout score. Then, all score were judged by minMax tree to found
out best choose position. To speed up minMax tree, alpha-beta punching was used
to reduce the number of node to expand.

\subsection{Representation}
Some main data are maintain during process: color, candidate list, chessboard.
Others data would be specified inside functions.

\begin{itemize}
  \item \textbf{color}: The chess color of current AI
  \item \textbf{candidate\_list}: Candidate point coordination, nested list.
  Final tuple in list would be return as final choice.
  \item \textbf{chessboard}: The chessboard  layout from player.
  \item \textbf{depth}: The search depth
\end{itemize}


\subsection{Architecture}
Here list all functions in given class AI:
\begin{itemize}
  \item Given function:
  \begin{itemize}
    \item \textbf{\_\_init\_\_}: initial function, including global variables.
    \item \textbf{first\_chess}: if AI is black chess, this function run.
    \item \textbf{go}: receive current chessboard layout, and return AI's choice.
  \end{itemize}
  \item Self-defined function:
    \begin{itemize}
      \item \textbf{alphaBeta}: minMax tree and alpha-beta punching.
      \item \textbf{genNext}: generate available position on chessboard.
      \item \textbf{evaluateState}: evaluate current chessboard score of black
	      or white
      \item \textbf{evaluateLine}: evaluate current line score of black or white
      \item \textbf{mapValue}: map different chess layout in single line to score
    \end{itemize}
\end{itemize}
The class AI would be executed in test platform.


\subsection{Detail of Algorithm}
Here describes some vital functions.
\begin{itemize}
    \item \textbf{alphaBeta}: use minMax tree with alpha-beta punching to choose
	    best position
    \begin{algorithm}[H]
     \caption{alphaBeta}
     \begin{algorithmic}[1]
     \renewcommand{\algorithmicrequire}{\textbf{Input:}}
     \renewcommand{\algorithmicensure}{\textbf{Output:}}
     \REQUIRE $source\_chessboard, depth, alpha, beta, chess\_color$
     \ENSURE  $max\_value$
     \IF {($depth == 0$)}
     \STATE \RETURN  evaluateState($source\_chessboard$)
     \ENDIF

     \STATE $child\_chessboard\leftarrow source\_chessboard$
     \STATE $cadidate\_points\leftarrow genNext(child\_chessboard)$
     \STATE $max\_value\leftarrow alpha$
     \STATE $max\_point\leftarrow candidate\_points[0]$ 

     \FOR {$point$ in  $candidate\_points$}
     \STATE $child\_chessboard[point]\leftarrow chess\_color$ \COMMENT{action}
     \STATE $value \leftarrow -alphaBeta(child\_chessboard, depth-1,-beta,-alpha, -chess\_color)$ \COMMENT{recursion}
     \STATE $child\_chessboard[point]\leftarrow 0$ \COMMENT{undo action}
     \IF {($value > max\_value$)}
     \STATE $max\_value \leftarrow value$
     \STATE $max\_point \leftarrow point$
        \IF {($max_value >= beta$)}
          \STATE break
        \ENDIF
     \ENDIF
     \ENDFOR
     \IF{(depth == initial\_depth)}
     \STATE $candidate\_list.append(max\_point)$ \COMMENT{add max value point to candidate list}
     \ENDIF
     \RETURN $max\_value$
     \end{algorithmic}
   \end{algorithm}

   \item \textbf{genNext}: generate most possible position
     \begin{algorithm}[H]
     \caption{genNext}
     \begin{algorithmic}[2]
     \renewcommand{\algorithmicrequire}{\textbf{Input:}}
     \renewcommand{\algorithmicensure}{\textbf{Output:}}
     \REQUIRE $chess\_chessboard, chess\_color$
     \ENSURE  $score$
     \STATE $pos \leftarrow empty position$
     \STATE $neighbor\_points \leftarrow []$ \COMMENT{initial}
     \FOR {$point$ in $pos$ }
       \IF{($point$ has any neighbor in 4 direction)}
       \STATE add $point$ to $neighbor\_points$
       \STATE break
       \ENDIF
     \ENDFOR
     \RETURN $neighbor\_points$
     \end{algorithmic}
     \end{algorithm}

  \item \textbf{evaluateState}: return current chessboard layout score
    \begin{algorithm}[H]
     \caption{evaluateState}
     \begin{algorithmic}[3]
     \renewcommand{\algorithmicrequire}{\textbf{Input:}}
     \renewcommand{\algorithmicensure}{\textbf{Output:}}
     \REQUIRE $chessboard, chess\_color$
     \ENSURE  $score$
     \STATE $score \leftarrow 0$ \COMMENT{Initial $score$}
     \FOR {every $line$ in horizontal ,vertical, diagonal}
       \STATE $score \leftarrow score + evaluateLine(line, chess\_color$)
     \ENDFOR
     \RETURN $score$
     \end{algorithmic}
     \end{algorithm}
 
   \item \textbf{evaluateLine}: get single line score
     \begin{algorithm}[H]
     \caption{evaluateLine}
     \begin{algorithmic}[4]
     \renewcommand{\algorithmicrequire}{\textbf{Input:}}
     \renewcommand{\algorithmicensure}{\textbf{Output:}}
     \REQUIRE $chess_line, chess\_color$
     \ENSURE  $score$
     \STATE $score \leftarrow 0$
     \STATE $continue\_point \leftarrow 1$
     \STATE $blcok\_point \leftarrow 0$
     \WHILE{$point$ in $line$}
       \COMMENT{found own chess}
       \IF{$point == chess\_color$} 
         \STATE $continue\_point \leftarrow 1$
         \STATE $block\_point \leftarrow 0$
         \IF{($point == -chess\_color$)}
           \STATE $block\_point \leftarrow block\_point+1$
         \ENDIF
         \STATE $point \leftarrow$ next $point$ in $line$
         \WHILE{$point$ still in $line$ \AND $point==chess\_color$}
           \STATE $point \leftarrow$ next $point$ in $line$
           \STATE $continue\_point \leftarrow continue\_point+1$
           \IF{$point == -chess\_color$ }
             \STATE $block\_point \leftarrow block\_point+1$
           \ENDIF
         \ENDWHILE
         \STATE $score \leftarrow score+ mapValue(continue\_point, bllock\_point)$
       \ENDIF
       \STATE $point \leftarrow$ next $point$ in $line$
     \ENDWHILE
     \RETURN $score$
     \end{algorithmic}
     \end{algorithm}

\end{itemize}


\section{Empirical Verification}
Empirical verification is compared with given test\_code.py and public test
platform.

\subsection{Design}
Successfully to evaluate chessboard score currently. When depth is two, program
output current answer within five second. When I try to run in depth of four,
right answer almost take half of hour. In a word, more powerful evalute
function and punching(heuristic function) are required.

\subsection{Data and data structure}
I use test data provided by code\_test.py to test. Numpy matrix are used widely.

\subsection{Performance}
Right output in depth of two and time of five second. It run so slow when
depth is four.

\subsection{Result}
Pass all test but fail to attend completion.

\subsection{Analysis}
The evaluate chessboard layout score cost linear time. minMax tree cost most
time. The deeper, the more time it cost.

\section*{Acknowledgment}
Thanks TA Yao Zhao who explain question and provide general method to solve it.
And I also thanks for Kebing Sun help me debug my algorithm mistake.

\bibliographystyle{IEEEtran}
\begin{thebibliography}{1}
\bibitem{1}
Wikipedia contributors, [Online], Available: https://en.wikipedia.org/wiki/Gomoku,
[Accessed: 31- Oct- 2018].
\end{thebibliography}

% that's all folks
\end{document}


