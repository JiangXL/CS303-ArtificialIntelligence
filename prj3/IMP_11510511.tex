\documentclass[conference,compsoc]{IEEEtran}

% *** CITATION PACKAGES ***
%
\ifCLASSOPTIONcompsoc
  % IEEE Computer Society needs nocompress option
  % requires cite.sty v4.0 or later (November 2003)
  \usepackage[nocompress]{cite}
\else
  % normal IEEE
  \usepackage{cite}
\fi
\usepackage{amsmath}
\usepackage{algorithm}
\usepackage{algorithmic}
\renewcommand{\algorithmicrequire}{\textbf{Input:}}
\renewcommand{\algorithmicensure}{\textbf{Output:}}
\usepackage{array}
\usepackage{url}
\hyphenation{op-tical net-works semi-conduc-tor}


\begin{document}
\title{Project3: IMP}

\author{\IEEEauthorblockN{Yuejian MO  11510511}
\IEEEauthorblockA{Department of Biology\\
Southern Unviersity of Science and Technology\\
Email: 11510511@mail.sustc.edu.cn}}
% make the title area
\maketitle

\IEEEpeerreviewmaketitle


\section{Preliminaries}

The conceptually simplest model of this type is what one could call the
$Independent Cascade Model$, investigated recently in the context of
marketing by reports. We again start with an initial set of active nodes
$A_0$, and the process unfolds in discrete steps according to the
following randomized rule. When node $v$ first becomes active in step $t$
,it is given a single chance to activate each currently inactive
neighbor $w$; it succeeds with a probability $p_{v,w}$, a parameter of
the system, independently of the history thus far. (If $w$ has multiple
newly activated neighbors, their attempts are sequenced in an arbitray 
order.) If $v$ succeeds, then $w$ will become active in step $t+1$; but
whether or not $v$ succeeds, then it cannnot make any further attempts
to activate $w$ in subsequent rounds. Again, the process runs until no
more activations are possible.

\subsection{Software}


\subsection{Algorithm}
   


\section{Methodology}

\subsection{Representation}



\subsection{Architecture}


\subsection{Detail of Algorithm}


\section{Empirical Verification}


\subsection{Design}

\subsection{Data and data structure}

\subsection{Performance}

\subsection{Result}

\subsection{Analysis}



\section*{Acknowledgment}

\bibliographystyle{IEEEtran}

\begin{thebibliography}{1}
\bibitem{reference} XXXXXXX
\end{thebibliography}

